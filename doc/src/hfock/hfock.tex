%%
%% Automatically generated file from DocOnce source
%% (https://github.com/hplgit/doconce/)
%%
%%


%-------------------- begin preamble ----------------------

\documentclass[%
oneside,                 % oneside: electronic viewing, twoside: printing
final,                   % or draft (marks overfull hboxes, figures with paths)
10pt]{article}

\listfiles               % print all files needed to compile this document

\usepackage{relsize,makeidx,color,setspace,amsmath,amsfonts}
\usepackage[table]{xcolor}
\usepackage{bm,microtype}

\usepackage[pdftex]{graphicx}

\usepackage[T1]{fontenc}
%\usepackage[latin1]{inputenc}
\usepackage{ucs}
\usepackage[utf8x]{inputenc}

\usepackage{lmodern}         % Latin Modern fonts derived from Computer Modern

% Hyperlinks in PDF:
\definecolor{linkcolor}{rgb}{0,0,0.4}
\usepackage{hyperref}
\hypersetup{
    breaklinks=true,
    colorlinks=true,
    linkcolor=linkcolor,
    urlcolor=linkcolor,
    citecolor=black,
    filecolor=black,
    %filecolor=blue,
    pdfmenubar=true,
    pdftoolbar=true,
    bookmarksdepth=3   % Uncomment (and tweak) for PDF bookmarks with more levels than the TOC
    }
%\hyperbaseurl{}   % hyperlinks are relative to this root

\setcounter{tocdepth}{2}  % number chapter, section, subsection

% prevent orhpans and widows
\clubpenalty = 10000
\widowpenalty = 10000

\newenvironment{doconceexercise}{}{}
\newcounter{doconceexercisecounter}
% --- begin definition of \listofexercises command ---
\makeatletter
\newcommand\listofexercises{\section*{List of Exercises}
\@starttoc{loe}
}
\newcommand*{\l@doconceexercise}{\@dottedtocline{0}{0pt}{6.5em}}
\makeatother
% --- end definition of \listofexercises command ---



% ------ header in subexercises ------
%\newcommand{\subex}[1]{\paragraph{#1}}
%\newcommand{\subex}[1]{\par\vspace{1.7mm}\noindent{\bf #1}\ \ }
\makeatletter
% 1.5ex is the spacing above the header, 0.5em the spacing after subex title
\newcommand\subex{\@startsection*{paragraph}{4}{\z@}%
                  {1.5ex\@plus1ex \@minus.2ex}%
                  {-0.5em}%
                  {\normalfont\normalsize\bfseries}}
\makeatother


% --- end of standard preamble for documents ---


% insert custom LaTeX commands...

\raggedbottom
\makeindex

%-------------------- end preamble ----------------------

\begin{document}

% endif for #ifdef PREAMBLE


% ------------------- main content ----------------------

% Slides for PHY981


% ----------------- title -------------------------

\thispagestyle{empty}

\begin{center}
{\LARGE\bf
\begin{spacing}{1.25}
Hartree-Fock methods
\end{spacing}
}
\end{center}

% ----------------- author(s) -------------------------

\begin{center}
{\bf \href{{http://computationalphysics.no}}{Morten Hjorth-Jensen}, National Superconducting Cyclotron Laboratory and Department of Physics and Astronomy, Michigan State University, East Lansing, MI 48824, USA {\&} Department of Physics, University of Oslo, Oslo, Norway${}^{}$} \\ [0mm]
\end{center}

\begin{center}
% List of all institutions:
\end{center}
    
% ----------------- end author(s) -------------------------

\begin{center} % date
July 6-24 2015
\end{center}

\vspace{1cm}


\tableofcontents


\vspace{1cm} % after toc




\subsection*{Why Hartree-Fock? Derivation of Hartree-Fock equations in coordinate space}

Hartree-Fock (HF) theory is an algorithm for finding an approximative expression for the ground state of a given Hamiltonian. The basic ingredients are
\begin{itemize}
  \item Define a single-particle basis $\{\psi_{\alpha}\}$ so that
\end{itemize}

\noindent
\[ 
\hat{h}^{\mathrm{HF}}\psi_{\alpha} = \varepsilon_{\alpha}\psi_{\alpha}
\]
with the Hartree-Fock Hamiltonian defined as
\[
\hat{h}^{\mathrm{HF}}=\hat{t}+\hat{u}_{\mathrm{ext}}+\hat{u}^{\mathrm{HF}}
\]
\begin{itemize}
  \item The term  $\hat{u}^{\mathrm{HF}}$ is a single-particle potential to be determined by the HF algorithm.

  \item The HF algorithm means to choose $\hat{u}^{\mathrm{HF}}$ in order to have 
\end{itemize}

\noindent
\[ \langle \hat{H} \rangle = E^{\mathrm{HF}}= \langle \Phi_0 | \hat{H}|\Phi_0 \rangle
\]
that is to find a local minimum with a Slater determinant $\Phi_0$ being the ansatz for the ground state. 
\begin{itemize}
  \item The variational principle ensures that $E^{\mathrm{HF}} \ge E_0$, with $E_0$ the exact ground state energy.
\end{itemize}

\noindent
We will show that the Hartree-Fock Hamiltonian $\hat{h}^{\mathrm{HF}}$ equals our definition of the operator $\hat{f}$ discussed in connection with the new definition of the normal-ordered Hamiltonian (see later lectures), that is we have, for a specific matrix element
\[
\langle p |\hat{h}^{\mathrm{HF}}| q \rangle =\langle p |\hat{f}| q \rangle=\langle p|\hat{t}+\hat{u}_{\mathrm{ext}}|q \rangle +\sum_{i\le F} \langle pi | \hat{V} | qi\rangle_{AS},
\]
meaning that
\[
\langle p|\hat{u}^{\mathrm{HF}}|q\rangle = \sum_{i\le F} \langle pi | \hat{V} | qi\rangle_{AS}.
\]
The so-called Hartree-Fock potential $\hat{u}^{\mathrm{HF}}$ brings an explicit medium dependence due to the summation over all single-particle states below the Fermi level $F$. It brings also in an explicit dependence on the two-body interaction (in nuclear physics we can also have complicated three- or higher-body forces). The two-body interaction, with its contribution from the other bystanding fermions, creates an effective mean field in which a given fermion moves, in addition to the external potential $\hat{u}_{\mathrm{ext}}$ which confines the motion of the fermion. For systems like nuclei, there is no external confining potential. Nuclei are examples of self-bound systems, where the binding arises due to the intrinsic nature of the strong force. For nuclear systems thus, there would be no external one-body potential in the Hartree-Fock Hamiltonian. 

\subsection*{Variational Calculus and Lagrangian Multipliers}

The calculus of variations involves 
problems where the quantity to be minimized or maximized is an integral. 

In the general case we have an integral of the type
\[ 
E[\Phi]= \int_a^b f(\Phi(x),\frac{\partial \Phi}{\partial x},x)dx,
\]
where $E$ is the quantity which is sought minimized or maximized.
The problem is that although $f$ is a function of the variables $\Phi$, $\partial \Phi/\partial x$ and $x$, the exact dependence of
$\Phi$ on $x$ is not known.  This means again that even though the integral has fixed limits $a$ and $b$, the path of integration is
not known. In our case the unknown quantities are the single-particle wave functions and we wish to choose an integration path which makes
the functional $E[\Phi]$ stationary. This means that we want to find minima, or maxima or saddle points. In physics we search normally for minima.
Our task is therefore to find the minimum of $E[\Phi]$ so that its variation $\delta E$ is zero  subject to specific
constraints. In our case the constraints appear as the integral which expresses the orthogonality of the  single-particle wave functions.
The constraints can be treated via the technique of Lagrangian multipliers

Let us specialize to the expectation value of the energy for one particle in three-dimensions.
This expectation value reads
\[
  E=\int dxdydz \psi^*(x,y,z) \hat{H} \psi(x,y,z),
\]
with the constraint
\[
 \int dxdydz \psi^*(x,y,z) \psi(x,y,z)=1,
\]
and a Hamiltonian
\[
\hat{H}=-\frac{1}{2}\nabla^2+V(x,y,z).
\]
We will, for the sake of notational convenience,  skip the variables $x,y,z$ below, and write for example $V(x,y,z)=V$.

The integral involving the kinetic energy can be written as, with the function $\psi$ vanishing
strongly for large values of $x,y,z$ (given here by the limits $a$ and $b$), 
 \[
  \int_a^b dxdydz \psi^* \left(-\frac{1}{2}\nabla^2\right) \psi dxdydz = \psi^*\nabla\psi|_a^b+\int_a^b dxdydz\frac{1}{2}\nabla\psi^*\nabla\psi.
\]
We will drop the limits $a$ and $b$ in the remaining discussion. 
Inserting this expression into the expectation value for the energy and taking the variational minimum  we obtain
\[
\delta E = \delta \left\{\int dxdydz\left( \frac{1}{2}\nabla\psi^*\nabla\psi+V\psi^*\psi\right)\right\} = 0.
\]

The constraint appears in integral form as 
\[
 \int dxdydz \psi^* \psi=\mathrm{constant},
\]
and multiplying with a Lagrangian multiplier $\lambda$ and taking the variational minimum we obtain the final variational equation
\[
\delta \left\{\int dxdydz\left( \frac{1}{2}\nabla\psi^*\nabla\psi+V\psi^*\psi-\lambda\psi^*\psi\right)\right\} = 0.
\]
We introduce the function  $f$
\[
  f =  \frac{1}{2}\nabla\psi^*\nabla\psi+V\psi^*\psi-\lambda\psi^*\psi=
\frac{1}{2}(\psi^*_x\psi_x+\psi^*_y\psi_y+\psi^*_z\psi_z)+V\psi^*\psi-\lambda\psi^*\psi,
\]
where we have skipped the dependence on $x,y,z$ and introduced the shorthand $\psi_x$, $\psi_y$ and $\psi_z$  for the various derivatives.

For $\psi^*$ the Euler-Lagrange  equations yield
\[
\frac{\partial f}{\partial \psi^*}- \frac{\partial }{\partial x}\frac{\partial f}{\partial \psi^*_x}-\frac{\partial }{\partial y}\frac{\partial f}{\partial \psi^*_y}-\frac{\partial }{\partial z}\frac{\partial f}{\partial \psi^*_z}=0,
\] 
which results in 
\[
    -\frac{1}{2}(\psi_{xx}+\psi_{yy}+\psi_{zz})+V\psi=\lambda \psi.
\]
We can then identify the  Lagrangian multiplier as the energy of the system. The last equation is 
nothing but the standard 
Schroedinger equation and the variational  approach discussed here provides 
a powerful method for obtaining approximate solutions of the wave function.

\subsection*{Definitions and notations}

Before we proceed we need some definitions.
We will assume that the interacting part of the Hamiltonian
can be approximated by a two-body interaction.
This means that our Hamiltonian is written as the sum of some onebody part and a twobody part
\begin{equation}
    \hat{H} = \hat{H}_0 + \hat{H}_I 
    = \sum_{i=1}^A \hat{h}_0(x_i) + \sum_{i < j}^A \hat{v}(r_{ij}),
\label{Hnuclei}
\end{equation}
with 
\begin{equation}
  H_0=\sum_{i=1}^A \hat{h}_0(x_i).
\label{hinuclei}
\end{equation}
The onebody part $u_{\mathrm{ext}}(x_i)$ is normally approximated by a harmonic oscillator potential or the Coulomb interaction an electron feels from the nucleus. However, other potentials are fully possible, such as 
one derived from the self-consistent solution of the Hartree-Fock equations to be discussed here.



Our Hamiltonian is invariant under the permutation (interchange) of two particles.
Since we deal with fermions however, the total wave function is antisymmetric.
Let $\hat{P}$ be an operator which interchanges two particles.
Due to the symmetries we have ascribed to our Hamiltonian, this operator commutes with the total Hamiltonian,
\[
[\hat{H},\hat{P}] = 0,
 \]
meaning that $\Psi_{\lambda}(x_1, x_2, \dots , x_A)$ is an eigenfunction of 
$\hat{P}$ as well, that is
\[
\hat{P}_{ij}\Psi_{\lambda}(x_1, x_2, \dots,x_i,\dots,x_j,\dots,x_A)=
\beta\Psi_{\lambda}(x_1, x_2, \dots,x_i,\dots,x_j,\dots,x_A),
\]
where $\beta$ is the eigenvalue of $\hat{P}$. We have introduced the suffix $ij$ in order to indicate that we permute particles $i$ and $j$.
The Pauli principle tells us that the total wave function for a system of fermions
has to be antisymmetric, resulting in the eigenvalue $\beta = -1$.   



In our case we assume that  we can approximate the exact eigenfunction with a Slater determinant
\begin{equation}
   \Phi(x_1, x_2,\dots ,x_A,\alpha,\beta,\dots, \sigma)=\frac{1}{\sqrt{A!}}
\left| \begin{array}{ccccc} \psi_{\alpha}(x_1)& \psi_{\alpha}(x_2)& \dots & \dots & \psi_{\alpha}(x_A)\\
                            \psi_{\beta}(x_1)&\psi_{\beta}(x_2)& \dots & \dots & \psi_{\beta}(x_A)\\  
                            \dots & \dots & \dots & \dots & \dots \\
                            \dots & \dots & \dots & \dots & \dots \\
                     \psi_{\sigma}(x_1)&\psi_{\sigma}(x_2)& \dots & \dots & \psi_{\sigma}(x_A)\end{array} \right|, \label{eq:HartreeFockDet}
\end{equation}
where  $x_i$  stand for the coordinates and spin values of a particle $i$ and $\alpha,\beta,\dots, \gamma$ 
are quantum numbers needed to describe remaining quantum numbers.  




The single-particle function $\psi_{\alpha}(x_i)$  are eigenfunctions of the onebody
Hamiltonian $h_i$, that is
\[
\hat{h}_0(x_i)=\hat{t}(x_i) + \hat{u}_{\mathrm{ext}}(x_i),
\]
with eigenvalues 
\[
\hat{h}_0(x_i) \psi_{\alpha}(x_i)=\left(\hat{t}(x_i) + \hat{u}_{\mathrm{ext}}(x_i)\right)\psi_{\alpha}(x_i)=\varepsilon_{\alpha}\psi_{\alpha}(x_i).
\]
The energies $\varepsilon_{\alpha}$ are the so-called non-interacting single-particle energies, or unperturbed energies. 
The total energy is in this case the sum over all  single-particle energies, if no two-body or more complicated
many-body interactions are present.



Let us denote the ground state energy by $E_0$. According to the
variational principle we have
\[
  E_0 \le E[\Phi] = \int \Phi^*\hat{H}\Phi d\mathbf{\tau}
\]
where $\Phi$ is a trial function which we assume to be normalized
\[
  \int \Phi^*\Phi d\mathbf{\tau} = 1,
\]
where we have used the shorthand $d\mathbf{\tau}=dx_1dr_2\dots dr_A$.




In the Hartree-Fock method the trial function is the Slater
determinant of Eq.~(\ref{eq:HartreeFockDet}) which can be rewritten as 
\[
  \Phi(x_1,x_2,\dots,x_A,\alpha,\beta,\dots,\nu) = \frac{1}{\sqrt{A!}}\sum_{P} (-)^P\hat{P}\psi_{\alpha}(x_1)
    \psi_{\beta}(x_2)\dots\psi_{\nu}(x_A)=\sqrt{A!}\hat{A}\Phi_H,
\]
where we have introduced the antisymmetrization operator $\hat{A}$ defined by the 
summation over all possible permutations of two particles.



It is defined as
\begin{equation}
  \hat{A} = \frac{1}{A!}\sum_{p} (-)^p\hat{P},
\label{antiSymmetryOperator}
\end{equation}
with $p$ standing for the number of permutations. We have introduced for later use the so-called
Hartree-function, defined by the simple product of all possible single-particle functions
\[
  \Phi_H(x_1,x_2,\dots,x_A,\alpha,\beta,\dots,\nu) =
  \psi_{\alpha}(x_1)
    \psi_{\beta}(x_2)\dots\psi_{\nu}(x_A).
\]


Both $\hat{H}_0$ and $\hat{H}_I$ are invariant under all possible permutations of any two particles
and hence commute with $\hat{A}$
\begin{equation}
  [H_0,\hat{A}] = [H_I,\hat{A}] = 0. \label{commutionAntiSym}
\end{equation}
Furthermore, $\hat{A}$ satisfies
\begin{equation}
  \hat{A}^2 = \hat{A},  \label{AntiSymSquared}
\end{equation}
since every permutation of the Slater
determinant reproduces it. 



The expectation value of $\hat{H}_0$ 
\[
  \int \Phi^*\hat{H}_0\Phi d\mathbf{\tau} 
  = A! \int \Phi_H^*\hat{A}\hat{H}_0\hat{A}\Phi_H d\mathbf{\tau}
\]
is readily reduced to
\[
  \int \Phi^*\hat{H}_0\Phi d\mathbf{\tau} 
  = A! \int \Phi_H^*\hat{H}_0\hat{A}\Phi_H d\mathbf{\tau},
\]
where we have used Eqs.~(\ref{commutionAntiSym}) and
(\ref{AntiSymSquared}). The next step is to replace the antisymmetrization
operator by its definition and to
replace $\hat{H}_0$ with the sum of one-body operators
\[
  \int \Phi^*\hat{H}_0\Phi  d\mathbf{\tau}
  = \sum_{i=1}^A \sum_{p} (-)^p\int 
  \Phi_H^*\hat{h}_0\hat{P}\Phi_H d\mathbf{\tau}.
\]


The integral vanishes if two or more particles are permuted in only one
of the Hartree-functions $\Phi_H$ because the individual single-particle wave functions are
orthogonal. We obtain then
\[
  \int \Phi^*\hat{H}_0\Phi  d\mathbf{\tau}= \sum_{i=1}^A \int \Phi_H^*\hat{h}_0\Phi_H  d\mathbf{\tau}.
\]
Orthogonality of the single-particle functions allows us to further simplify the integral, and we
arrive at the following expression for the expectation values of the
sum of one-body Hamiltonians 
\begin{equation}
  \int \Phi^*\hat{H}_0\Phi  d\mathbf{\tau}
  = \sum_{\mu=1}^A \int \psi_{\mu}^*(x)\hat{h}_0\psi_{\mu}(x)dx
  d\mathbf{r}.
  \label{H1Expectation}
\end{equation}



We introduce the following shorthand for the above integral
\[
\langle \mu | \hat{h}_0 | \mu \rangle = \int \psi_{\mu}^*(x)\hat{h}_0\psi_{\mu}(x)dx,
\]
and rewrite Eq.~(\ref{H1Expectation}) as
\begin{equation}
  \int \Phi^*\hat{H}_0\Phi  d\tau
  = \sum_{\mu=1}^A \langle \mu | \hat{h}_0 | \mu \rangle.
  \label{H1Expectation1}
\end{equation}



The expectation value of the two-body part of the Hamiltonian is obtained in a
similar manner. We have
\[
  \int \Phi^*\hat{H}_I\Phi d\mathbf{\tau} 
  = A! \int \Phi_H^*\hat{A}\hat{H}_I\hat{A}\Phi_H d\mathbf{\tau},
\]
which reduces to
\[
 \int \Phi^*\hat{H}_I\Phi d\mathbf{\tau} 
  = \sum_{i\le j=1}^A \sum_{p} (-)^p\int 
  \Phi_H^*\hat{v}(r_{ij})\hat{P}\Phi_H d\mathbf{\tau},
\]
by following the same arguments as for the one-body
Hamiltonian. 



Because of the dependence on the inter-particle distance $r_{ij}$,  permutations of
any two particles no longer vanish, and we get
\[
  \int \Phi^*\hat{H}_I\Phi d\mathbf{\tau} 
  = \sum_{i < j=1}^A \int  
  \Phi_H^*\hat{v}(r_{ij})(1-P_{ij})\Phi_H d\mathbf{\tau}.
\]
where $P_{ij}$ is the permutation operator that interchanges
particle $i$ and particle $j$. Again we use the assumption that the single-particle wave functions
are orthogonal. 





We obtain
\begin{align}
  \int \Phi^*\hat{H}_I\Phi d\mathbf{\tau} 
  = \frac{1}{2}\sum_{\mu=1}^A\sum_{\nu=1}^A
    &\left[ \int \psi_{\mu}^*(x_i)\psi_{\nu}^*(x_j)\hat{v}(r_{ij})\psi_{\mu}(x_i)\psi_{\nu}(x_j)
    dx_idx_j \right.\\
  &\left.
  - \int \psi_{\mu}^*(x_i)\psi_{\nu}^*(x_j)
  \hat{v}(r_{ij})\psi_{\nu}(x_i)\psi_{\mu}(x_j)
  dx_idx_j
  \right]. \label{H2Expectation}
\end{align}
The first term is the so-called direct term. It is frequently also called the  Hartree term, 
while the second is due to the Pauli principle and is called
the exchange term or just the Fock term.
The factor  $1/2$ is introduced because we now run over
all pairs twice. 




The last equation allows us to  introduce some further definitions.  
The single-particle wave functions $\psi_{\mu}(x)$, defined by the quantum numbers $\mu$ and $x$
are defined as the overlap 
\[
   \psi_{\alpha}(x)  = \langle x | \alpha \rangle .
\]




We introduce the following shorthands for the above two integrals
\[
\langle \mu\nu|\hat{v}|\mu\nu\rangle =  \int \psi_{\mu}^*(x_i)\psi_{\nu}^*(x_j)\hat{v}(r_{ij})\psi_{\mu}(x_i)\psi_{\nu}(x_j)
    dx_idx_j,
\]
and
\[
\langle \mu\nu|\hat{v}|\nu\mu\rangle = \int \psi_{\mu}^*(x_i)\psi_{\nu}^*(x_j)
  \hat{v}(r_{ij})\psi_{\nu}(x_i)\psi_{\mu}(x_j)
  dx_idx_j.  
\]



\subsection*{Derivation of Hartree-Fock equations in coordinate space}

Let us denote the ground state energy by $E_0$. According to the
variational principle we have
\[
  E_0 \le E[\Phi] = \int \Phi^*\hat{H}\Phi d\mathbf{\tau}
\]
where $\Phi$ is a trial function which we assume to be normalized
\[
  \int \Phi^*\Phi d\mathbf{\tau} = 1,
\]
where we have used the shorthand $d\mathbf{\tau}=dx_1dx_2\dots dx_A$.




In the Hartree-Fock method the trial function is a Slater
determinant which can be rewritten as 
\[
  \Psi(x_1,x_2,\dots,x_A,\alpha,\beta,\dots,\nu) = \frac{1}{\sqrt{A!}}\sum_{P} (-)^PP\psi_{\alpha}(x_1)
    \psi_{\beta}(x_2)\dots\psi_{\nu}(x_A)=\sqrt{A!}\hat{A}\Phi_H,
\]
where we have introduced the anti-symmetrization operator $\hat{A}$ defined by the 
summation over all possible permutations \emph{p} of two fermions.
It is defined as
\[
  \hat{A} = \frac{1}{A!}\sum_{p} (-)^p\hat{P},
\]
with the the Hartree-function given by the simple product of all possible single-particle function
\[
  \Phi_H(x_1,x_2,\dots,x_A,\alpha,\beta,\dots,\nu) =
  \psi_{\alpha}(x_1)
    \psi_{\beta}(x_2)\dots\psi_{\nu}(x_A).
\]



Our functional is written as
\[
  E[\Phi] = \sum_{\mu=1}^A \int \psi_{\mu}^*(x_i)\hat{h}_0(x_i)\psi_{\mu}(x_i) dx_i 
  + \frac{1}{2}\sum_{\mu=1}^A\sum_{\nu=1}^A
   \left[ \int \psi_{\mu}^*(x_i)\psi_{\nu}^*(x_j)\hat{v}(r_{ij})\psi_{\mu}(x_i)\psi_{\nu}(x_j)dx_idx_j- \int \psi_{\mu}^*(x_i)\psi_{\nu}^*(x_j)
 \hat{v}(r_{ij})\psi_{\nu}(x_i)\psi_{\mu}(x_j)dx_idx_j\right]
\]
The more compact version reads
\[
  E[\Phi] 
  = \sum_{\mu}^A \langle \mu | \hat{h}_0 | \mu\rangle+ \frac{1}{2}\sum_{\mu\nu}^A\left[\langle \mu\nu |\hat{v}|\mu\nu\rangle-\langle \nu\mu |\hat{v}|\mu\nu\rangle\right].
\]



Since the interaction is invariant under the interchange of two particles it means for example that we have
\[
\langle \mu\nu|\hat{v}|\mu\nu\rangle =  \langle \nu\mu|\hat{v}|\nu\mu\rangle,  
\]
or in the more general case
\[
\langle \mu\nu|\hat{v}|\sigma\tau\rangle =  \langle \nu\mu|\hat{v}|\tau\sigma\rangle.  
\]


The direct and exchange matrix elements can be  brought together if we define the antisymmetrized matrix element
\[
\langle \mu\nu|\hat{v}|\mu\nu\rangle_{AS}= \langle \mu\nu|\hat{v}|\mu\nu\rangle-\langle \mu\nu|\hat{v}|\nu\mu\rangle,
\]
or for a general matrix element  
\[
\langle \mu\nu|\hat{v}|\sigma\tau\rangle_{AS}= \langle \mu\nu|\hat{v}|\sigma\tau\rangle-\langle \mu\nu|\hat{v}|\tau\sigma\rangle.
\]
It has the symmetry property
\[
\langle \mu\nu|\hat{v}|\sigma\tau\rangle_{AS}= -\langle \mu\nu|\hat{v}|\tau\sigma\rangle_{AS}=-\langle \nu\mu|\hat{v}|\sigma\tau\rangle_{AS}.
\]
The antisymmetric matrix element is also hermitian, implying 
\[
\langle \mu\nu|\hat{v}|\sigma\tau\rangle_{AS}= \langle \sigma\tau|\hat{v}|\mu\nu\rangle_{AS}.
\]




With these notations we rewrite the Hartree-Fock functional as
\begin{equation}
  \int \Phi^*\hat{H_I}\Phi d\mathbf{\tau} 
  = \frac{1}{2}\sum_{\mu=1}^A\sum_{\nu=1}^A \langle \mu\nu|\hat{v}|\mu\nu\rangle_{AS}. \label{H2Expectation2}
\end{equation}

Adding the contribution from the one-body operator $\hat{H}_0$ to
(\ref{H2Expectation2}) we obtain the energy functional 
\begin{equation}
  E[\Phi] 
  = \sum_{\mu=1}^A \langle \mu | h | \mu \rangle +
  \frac{1}{2}\sum_{{\mu}=1}^A\sum_{{\nu}=1}^A \langle \mu\nu|\hat{v}|\mu\nu\rangle_{AS}. \label{FunctionalEPhi}
\end{equation}
In our coordinate space derivations below we will spell out the Hartree-Fock equations in terms of their integrals.




If we generalize the Euler-Lagrange equations to more variables 
and introduce $N^2$ Lagrange multipliers which we denote by 
$\epsilon_{\mu\nu}$, we can write the variational equation for the functional of $E$
\[
  \delta E - \sum_{\mu\nu}^A \epsilon_{\mu\nu} \delta
  \int \psi_{\mu}^* \psi_{\nu} = 0.
\]
For the orthogonal wave functions $\psi_{i}$ this reduces to
\[
  \delta E - \sum_{\mu=1}^A \epsilon_{\mu} \delta
  \int \psi_{\mu}^* \psi_{\mu} = 0.
\]




Variation with respect to the single-particle wave functions $\psi_{\mu}$ yields then
\[
  \sum_{\mu=1}^A \int \delta\psi_{\mu}^*\hat{h_0}(x_i)\psi_{\mu}
  dx_i  
  + \frac{1}{2}\sum_{{\mu}=1}^A\sum_{{\nu}=1}^A \left[ \int
  \delta\psi_{\mu}^*\psi_{\nu}^*\hat{v}(r_{ij})\psi_{\mu}\psi_{\nu} dx_idx_j- \int
  \delta\psi_{\mu}^*\psi_{\nu}^*\hat{v}(r_{ij})\psi_{\nu}\psi_{\mu}
  dx_idx_j \right]+ 
\]
\[
\sum_{\mu=1}^A \int \psi_{\mu}^*\hat{h_0}(x_i)\delta\psi_{\mu}
  dx_i 
  + \frac{1}{2}\sum_{{\mu}=1}^A\sum_{{\nu}=1}^A \left[ \int
  \psi_{\mu}^*\psi_{\nu}^*\hat{v}(r_{ij})\delta\psi_{\mu}\psi_{\nu} dx_idx_j- \int
  \psi_{\mu}^*\psi_{\nu}^*\hat{v}(r_{ij})\psi_{\nu}\delta\psi_{\mu}
  dx_idx_j \right]-  \sum_{{\mu}=1}^A E_{\mu} \int \delta\psi_{\mu}^*
  \psi_{\mu}dx_i
  -  \sum_{{\mu}=1}^A E_{\mu} \int \psi_{\mu}^*
  \delta\psi_{\mu}dx_i = 0.
\]




Although the variations $\delta\psi$ and $\delta\psi^*$ are not
independent, they may in fact be treated as such, so that the 
terms dependent on either $\delta\psi$ and $\delta\psi^*$ individually 
may be set equal to zero. To see this, simply 
replace the arbitrary variation $\delta\psi$ by $i\delta\psi$, so that
$\delta\psi^*$ is replaced by $-i\delta\psi^*$, and combine the two
equations. We thus arrive at the Hartree-Fock equations
\begin{equation}
\left[ -\frac{1}{2}\nabla_i^2+ \sum_{\nu=1}^A\int \psi_{\nu}^*(x_j)\hat{v}(r_{ij})\psi_{\nu}(x_j)dx_j \right]\psi_{\mu}(x_i) - \left[ \sum_{{\nu}=1}^A \int\psi_{\nu}^*(x_j)\hat{v}(r_{ij})\psi_{\mu}(x_j) dx_j\right] \psi_{\nu}(x_i) = \epsilon_{\mu} \psi_{\mu}(x_i).  \label{eq:hartreefockcoordinatespace}
\end{equation}
Notice that the integration $\int dx_j$ implies an
integration over the spatial coordinates $\mathbf{r_j}$ and a summation
over the spin-coordinate of fermion $j$. We note that the factor of $1/2$ in front of the sum involving the two-body interaction, has been removed. This is due to the fact that we need to vary both $\delta\psi_{\mu}^*$ and
$\delta\psi_{\nu}^*$. Using the symmetry properties of the two-body interaction and interchanging $\mu$ and $\nu$
as summation indices, we obtain two identical terms. 




The two first terms in the last equation are the one-body kinetic energy and the
electron-nucleus potential. The third or \emph{direct} term is the averaged electronic repulsion of the other
electrons. As written, the
term includes the \emph{self-interaction} of 
electrons when $\mu=\nu$. The self-interaction is cancelled in the fourth
term, or the \emph{exchange} term. The exchange term results from our
inclusion of the Pauli principle and the assumed determinantal form of
the wave-function. Equation (\ref{eq:hartreefockcoordinatespace}), in addition to the kinetic energy and the attraction from the atomic nucleus that confines the motion of a single electron,   represents now the motion of a single-particle modified by the two-body interaction. The additional contribution to the Schroedinger equation due to the two-body interaction, represents a mean field set up by all the other bystanding electrons, the latter given by the sum over all single-particle states occupied by $N$ electrons. 

The Hartree-Fock equation is an example of an integro-differential equation. These equations involve repeated calculations of integrals, in addition to the solution of a set of coupled differential equations. 
The Hartree-Fock equations can also be rewritten in terms of an eigenvalue problem. The solution of an eigenvalue problem represents often a more practical algorithm and the  solution of  coupled  integro-differential equations.
This alternative derivation of the Hartree-Fock equations is given below.




\subsection*{Analysis of Hartree-Fock equations in coordinate space}

  A theoretically convenient form of the
Hartree-Fock equation is to regard the direct and exchange operator
defined through 
\begin{equation*}
  V_{\mu}^{d}(x_i) = \int \psi_{\mu}^*(x_j) 
 \hat{v}(r_{ij})\psi_{\mu}(x_j) dx_j
\end{equation*}
and
\begin{equation*}
  V_{\mu}^{ex}(x_i) g(x_i) 
  = \left(\int \psi_{\mu}^*(x_j) 
 \hat{v}(r_{ij})g(x_j) dx_j
  \right)\psi_{\mu}(x_i),
\end{equation*}
respectively. 




The function $g(x_i)$ is an arbitrary function,
and by the substitution $g(x_i) = \psi_{\nu}(x_i)$
we get
\begin{equation*}
  V_{\mu}^{ex}(x_i) \psi_{\nu}(x_i) 
  = \left(\int \psi_{\mu}^*(x_j) 
 \hat{v}(r_{ij})\psi_{\nu}(x_j)
  dx_j\right)\psi_{\mu}(x_i).
\end{equation*}
We may then rewrite the Hartree-Fock equations as
\[
  \hat{h}^{HF}(x_i) \psi_{\nu}(x_i) = \epsilon_{\nu}\psi_{\nu}(x_i),
\]
with
\[
  \hat{h}^{HF}(x_i)= \hat{h}_0(x_i) + \sum_{\mu=1}^AV_{\mu}^{d}(x_i) -
  \sum_{\mu=1}^AV_{\mu}^{ex}(x_i),
\]
and where $\hat{h}_0(i)$ is the one-body part. The latter is normally chosen as a part which yields solutions in closed form. The harmonic oscilltor is a classical problem thereof.
We normally rewrite the last equation as
\[
  \hat{h}^{HF}(x_i)= \hat{h}_0(x_i) + \hat{u}^{HF}(x_i). 
\]




\subsection*{Hartree-Fock by varying the coefficients of a wave function expansion}

Another possibility is to expand the single-particle functions in a known basis  and vary the coefficients, 
that is, the new single-particle wave function is written as a linear expansion
in terms of a fixed chosen orthogonal basis (for example the well-known harmonic oscillator functions or the hydrogen-like functions etc).
We define our new Hartree-Fock single-particle basis by performing a unitary transformation 
on our previous basis (labelled with greek indices) as
\begin{equation}
\psi_p^{HF}  = \sum_{\lambda} C_{p\lambda}\phi_{\lambda}. \label{eq:newbasis}
\end{equation}
In this case we vary the coefficients $C_{p\lambda}$. If the basis has infinitely many solutions, we need
to truncate the above sum.  We assume that the basis $\phi_{\lambda}$ is orthogonal. A unitary transformation keeps the orthogonality, as discussed in exercise 1 below.  




It is normal to choose a single-particle basis defined as the eigenfunctions
of parts of the full Hamiltonian. The typical situation consists of the solutions of the one-body part of the Hamiltonian, that is we have
\[
\hat{h}_0\phi_{\lambda}=\epsilon_{\lambda}\phi_{\lambda}.
\]
The single-particle wave functions $\phi_{\lambda}({\bf r})$, defined by the quantum numbers $\lambda$ and ${\bf r}$
are defined as the overlap 
\[
   \phi_{\lambda}({\bf r})  = \langle {\bf r} | \lambda \rangle .
\]




In our discussions hereafter we will use our definitions of single-particle states above and below the Fermi ($F$) level given by the labels
$ijkl\dots \le F$ for so-called single-hole states and $abcd\dots > F$ for so-called particle states.
For general single-particle states we employ the labels $pqrs\dots$. 




In Eq.~(\ref{FunctionalEPhi}), restated here
\[
  E[\Phi] 
  = \sum_{\mu=1}^A \langle \mu | h | \mu \rangle +
  \frac{1}{2}\sum_{{\mu}=1}^A\sum_{{\nu}=1}^A \langle \mu\nu|\hat{v}|\mu\nu\rangle_{AS},
\]
we found the expression for the energy functional in terms of the basis function $\phi_{\lambda}({\bf r})$. We then  varied the above energy functional with respect to the basis functions $|\mu \rangle$. 
Now we are interested in defining a new basis defined in terms of
a chosen basis as defined in Eq.~(\ref{eq:newbasis}). We can then rewrite the energy functional as
\begin{equation}
  E[\Phi^{HF}] 
  = \sum_{i=1}^A \langle i | h | i \rangle +
  \frac{1}{2}\sum_{ij=1}^A\langle ij|\hat{v}|ij\rangle_{AS}, \label{FunctionalEPhi2}
\end{equation}
where $\Phi^{HF}$ is the new Slater determinant defined by the new basis of Eq.~(\ref{eq:newbasis}). 





Using Eq.~(\ref{eq:newbasis}) we can rewrite Eq.~(\ref{FunctionalEPhi2}) as 
\begin{equation}
  E[\Psi] 
  = \sum_{i=1}^A \sum_{\alpha\beta} C^*_{i\alpha}C_{i\beta}\langle \alpha | h | \beta \rangle +
  \frac{1}{2}\sum_{ij=1}^A\sum_{{\alpha\beta\gamma\delta}} C^*_{i\alpha}C^*_{j\beta}C_{i\gamma}C_{j\delta}\langle \alpha\beta|\hat{v}|\gamma\delta\rangle_{AS}. \label{FunctionalEPhi3}
\end{equation}





We wish now to minimize the above functional. We introduce again a set of Lagrange multipliers, noting that
since $\langle i | j \rangle = \delta_{i,j}$ and $\langle \alpha | \beta \rangle = \delta_{\alpha,\beta}$, 
the coefficients $C_{i\gamma}$ obey the relation
\[
 \langle i | j \rangle=\delta_{i,j}=\sum_{\alpha\beta} C^*_{i\alpha}C_{i\beta}\langle \alpha | \beta \rangle=
\sum_{\alpha} C^*_{i\alpha}C_{i\alpha},
\]
which allows us to define a functional to be minimized that reads
\begin{equation}
  F[\Phi^{HF}]=E[\Phi^{HF}] - \sum_{i=1}^A\epsilon_i\sum_{\alpha} C^*_{i\alpha}C_{i\alpha}.
\end{equation}







Minimizing with respect to $C^*_{i\alpha}$, remembering that the equations for $C^*_{i\alpha}$ and $C_{i\alpha}$
can be written as two  independent equations, we obtain
\[
\frac{d}{dC^*_{i\alpha}}\left[  E[\Phi^{HF}] - \sum_{j}\epsilon_j\sum_{\alpha} C^*_{j\alpha}C_{j\alpha}\right]=0,
\]
which yields for every single-particle state $i$ and index $\alpha$ (recalling that the coefficients $C_{i\alpha}$ are matrix elements of a unitary (or orthogonal for a real symmetric matrix) matrix)
the following Hartree-Fock equations
\[
\sum_{\beta} C_{i\beta}\langle \alpha | h | \beta \rangle+
\sum_{j=1}^A\sum_{\beta\gamma\delta} C^*_{j\beta}C_{j\delta}C_{i\gamma}\langle \alpha\beta|\hat{v}|\gamma\delta\rangle_{AS}=\epsilon_i^{HF}C_{i\alpha}.
\]





We can rewrite this equation as (changing dummy variables)
\[
\sum_{\beta} \left\{\langle \alpha | h | \beta \rangle+
\sum_{j}^A\sum_{\gamma\delta} C^*_{j\gamma}C_{j\delta}\langle \alpha\gamma|\hat{v}|\beta\delta\rangle_{AS}\right\}C_{i\beta}=\epsilon_i^{HF}C_{i\alpha}.
\]
Note that the sums over greek indices run over the number of basis set functions (in principle an infinite number).





Defining 
\[
h_{\alpha\beta}^{HF}=\langle \alpha | h | \beta \rangle+
\sum_{j=1}^A\sum_{\gamma\delta} C^*_{j\gamma}C_{j\delta}\langle \alpha\gamma|\hat{v}|\beta\delta\rangle_{AS},
\]
we can rewrite the new equations as 
\begin{equation}
\sum_{\gamma}h_{\alpha\beta}^{HF}C_{i\beta}=\epsilon_i^{HF}C_{i\alpha}. \label{eq:newhf}
\end{equation}
The latter is nothing but a standard eigenvalue problem. Compared with Eq.~(\ref{eq:hartreefockcoordinatespace}),
we see that we do not need to compute any integrals in an iterative procedure for solving the equations.
It suffices to tabulate the matrix elements $\langle \alpha | h | \beta \rangle$ and $\langle \alpha\gamma|\hat{v}|\beta\delta\rangle_{AS}$ once and for all. Successive iterations require thus only a look-up in tables over one-body and two-body matrix elements. These details will be discussed below when we solve the Hartree-Fock equations numerical. 



\subsection*{Hartree-Fock algorithm}

Our Hartree-Fock matrix  is thus
\[
\hat{h}_{\alpha\beta}^{HF}=\langle \alpha | \hat{h}_0 | \beta \rangle+
\sum_{j=1}^A\sum_{\gamma\delta} C^*_{j\gamma}C_{j\delta}\langle \alpha\gamma|\hat{v}|\beta\delta\rangle_{AS}.
\]
The Hartree-Fock equations are solved in an iterative waym starting with a guess for the coefficients $C_{j\gamma}=\delta_{j,\gamma}$ and solving the equations by diagonalization till the new single-particle energies
$\epsilon_i^{\mathrm{HF}}$ do not change anymore by a prefixed quantity. 




Normally we assume that the single-particle basis $|\beta\rangle$ forms an eigenbasis for the operator
$\hat{h}_0$, meaning that the Hartree-Fock matrix becomes  
\[
\hat{h}_{\alpha\beta}^{HF}=\epsilon_{\alpha}\delta_{\alpha,\beta}+
\sum_{j=1}^A\sum_{\gamma\delta} C^*_{j\gamma}C_{j\delta}\langle \alpha\gamma|\hat{v}|\beta\delta\rangle_{AS}.
\]
The Hartree-Fock eigenvalue problem
\[
\sum_{\beta}\hat{h}_{\alpha\beta}^{HF}C_{i\beta}=\epsilon_i^{\mathrm{HF}}C_{i\alpha},
\]
can be written out in a more compact form as
\[
\hat{h}^{HF}\hat{C}=\epsilon^{\mathrm{HF}}\hat{C}. 
\]




The Hartree-Fock equations are, in their simplest form, solved in an iterative way, starting with a guess for the
coefficients $C_{i\alpha}$. We label the coefficients as $C_{i\alpha}^{(n)}$, where the subscript $n$ stands for iteration $n$.
To set up the algorithm we can proceed as follows:

\begin{itemize}
 \item We start with a guess $C_{i\alpha}^{(0)}=\delta_{i,\alpha}$. Alternatively, we could have used random starting values as long as the vectors are normalized. Another possibility is to give states below the Fermi level a larger weight.

 \item The Hartree-Fock matrix simplifies then to (assuming that the coefficients $C_{i\alpha} $  are real)
\end{itemize}

\noindent
\[
\hat{h}_{\alpha\beta}^{HF}=\epsilon_{\alpha}\delta_{\alpha,\beta}+
\sum_{j = 1}^A\sum_{\gamma\delta} C_{j\gamma}^{(0)}C_{j\delta}^{(0)}\langle \alpha\gamma|\hat{v}|\beta\delta\rangle_{AS}.
\]




Solving the Hartree-Fock eigenvalue problem yields then new eigenvectors $C_{i\alpha}^{(1)}$ and eigenvalues
$\epsilon_i^{HF(1)}$. 
\begin{itemize}
 \item With the new eigenvalues we can set up a new Hartree-Fock potential 
\end{itemize}

\noindent
\[
\sum_{j = 1}^A\sum_{\gamma\delta} C_{j\gamma}^{(1)}C_{j\delta}^{(1)}\langle \alpha\gamma|\hat{v}|\beta\delta\rangle_{AS}.
\]
The diagonalization with the new Hartree-Fock potential yields new eigenvectors and eigenvalues.
This process is continued till for example
\[
\frac{\sum_{p} |\epsilon_i^{(n)}-\epsilon_i^{(n-1)}|}{m} \le \lambda,  
\]
where $\lambda$ is a user prefixed quantity ($\lambda \sim 10^{-8}$ or smaller) and $p$ runs over all calculated single-particle
energies and $m$ is the number of single-particle states.

\begin{itemize}
\item TODO: add code with Hartree-Fock for nuclear system
\end{itemize}

\noindent
\subsection*{Analysis of Hartree-Fock equations and Koopman's theorem}

We can rewrite the ground state energy by adding and subtracting $\hat{u}^{HF}(x_i)$ 
\[
  E_0^{HF} =\langle \Phi_0 | \hat{H} | \Phi_0\rangle = 
\sum_{i\le F}^A \langle i | \hat{h}_0 +\hat{u}^{HF}| j\rangle+ \frac{1}{2}\sum_{i\le F}^A\sum_{j \le F}^A\left[\langle ij |\hat{v}|ij \rangle-\langle ij|\hat{v}|ji\rangle\right]-\sum_{i\le F}^A \langle i |\hat{u}^{HF}| i\rangle,
\]
which results in
\[
  E_0^{HF}
  = \sum_{i\le F}^A \varepsilon_i^{HF} + \frac{1}{2}\sum_{i\le F}^A\sum_{j \le F}^A\left[\langle ij |\hat{v}|ij \rangle-\langle ij|\hat{v}|ji\rangle\right]-\sum_{i\le F}^A \langle i |\hat{u}^{HF}| i\rangle.
\]
Our single-particle states $ijk\dots$ are now single-particle states obtained from the solution of the Hartree-Fock equations.



Using our definition of the Hartree-Fock single-particle energies we obtain then the following expression for the total ground-state energy
\[
  E_0^{HF}
  = \sum_{i\le F}^A \varepsilon_i - \frac{1}{2}\sum_{i\le F}^A\sum_{j \le F}^A\left[\langle ij |\hat{v}|ij \rangle-\langle ij|\hat{v}|ji\rangle\right].
\]
This form will be used in our discussion of Koopman's theorem.



In the   atomic physics case we have 
\[
  E[\Phi^{\mathrm{HF}}(N)] 
  = \sum_{i=1}^H \langle i | \hat{h}_0 | i \rangle +
  \frac{1}{2}\sum_{ij=1}^N\langle ij|\hat{v}|ij\rangle_{AS},
\]
where $\Phi^{\mathrm{HF}}(N)$ is the new Slater determinant defined by the new basis of Eq.~(\ref{eq:newbasis})
for $N$ electrons (same $Z$).  If we assume that the single-particle wave functions in the new basis do not change 
when we remove one electron or add one electron, we can then define the corresponding energy for the $N-1$ systems as 
\[
  E[\Phi^{\mathrm{HF}}(N-1)] 
  = \sum_{i=1; i\ne k}^N \langle i | \hat{h}_0 | i \rangle +
  \frac{1}{2}\sum_{ij=1;i,j\ne k}^N\langle ij|\hat{v}|ij\rangle_{AS},
\]
where we have removed a single-particle state $k\le F$, that is a state below the Fermi level.  



Calculating the difference 
\[
  E[\Phi^{\mathrm{HF}}(N)]-   E[\Phi^{\mathrm{HF}}(N-1)] = \langle k | \hat{h}_0 | k \rangle +
  \frac{1}{2}\sum_{i=1;i\ne k}^N\langle ik|\hat{v}|ik\rangle_{AS} + \frac{1}{2}\sum_{j=1;j\ne k}^N\langle kj|\hat{v}|kj\rangle_{AS},
\]
we obtain
\[
  E[\Phi^{\mathrm{HF}}(N)]-   E[\Phi^{\mathrm{HF}}(N-1)] = \langle k | \hat{h}_0 | k \rangle +\sum_{j=1}^N\langle kj|\hat{v}|kj\rangle_{AS}
\]
which is just our definition of the Hartree-Fock single-particle energy
\[
  E[\Phi^{\mathrm{HF}}(N)]-   E[\Phi^{\mathrm{HF}}(N-1)] = \epsilon_k^{\mathrm{HF}} 
\]



Similarly, we can now compute the difference (we label the single-particle states above the Fermi level as $abcd > F$)
\[
  E[\Phi^{\mathrm{HF}}(N+1)]-   E[\Phi^{\mathrm{HF}}(N)]= \epsilon_a^{\mathrm{HF}}. 
\]
These two equations can thus be used to the electron affinity or ionization energies, respectively. 
Koopman's theorem states that for example the ionization energy of a closed-shell system is given by the energy of the highest occupied single-particle state.  If we assume that changing the number of electrons from $N$ to $N+1$ does not change the Hartree-Fock single-particle energies and eigenfunctions, then Koopman's theorem simply states that the ionization energy of an atom is given by the single-particle energy of the last bound state. In a similar way, we can also define the electron affinities. 




As an example, consider a simple model for atomic sodium, Na. Neutral sodium has eleven electrons, 
with the weakest bound one being confined the $3s$ single-particle quantum numbers. The energy needed to remove an electron from neutral sodium is rather small, 5.1391 eV, a feature which pertains to all alkali metals.
Having performed a  Hartree-Fock calculation for neutral sodium would then allows us to compute the
ionization energy by using the single-particle energy for the $3s$ states, namely $\epsilon_{3s}^{\mathrm{HF}}$. 

From these considerations, we see that Hartree-Fock theory allows us to make a connection between experimental 
observables (here ionization and affinity energies) and the underlying interactions between particles.  
In this sense, we are now linking the dynamics and structure of a many-body system with the laws of motion which govern the system. Our approach is a reductionistic one, meaning that we want to understand the laws of motion 
in terms of the particles or degrees of freedom which we believe are the fundamental ones. Our Slater determinant, being constructed as the product of various single-particle functions, follows this philosophy.




With similar arguments as in atomic physics, we can now use Hartree-Fock theory to make a link
between nuclear forces and separation energies. Changing to nuclear system, we define
\[
  E[\Phi^{\mathrm{HF}}(A)] 
  = \sum_{i=1}^A \langle i | \hat{h}_0 | i \rangle +
  \frac{1}{2}\sum_{ij=1}^A\langle ij|\hat{v}|ij\rangle_{AS},
\]
where $\Phi^{\mathrm{HF}}(A)$ is the new Slater determinant defined by the new basis of Eq.~(\ref{eq:newbasis})
for $A$ nucleons, where $A=N+Z$, with $N$ now being the number of neutrons and $Z$ th enumber of protons.  If we assume again that the single-particle wave functions in the new basis do not change from a nucleus with $A$ nucleons to a nucleus with $A-1$  nucleons, we can then define the corresponding energy for the $A-1$ systems as 
\[
  E[\Phi^{\mathrm{HF}}(A-1)] 
  = \sum_{i=1; i\ne k}^A \langle i | \hat{h}_0 | i \rangle +
  \frac{1}{2}\sum_{ij=1;i,j\ne k}^A\langle ij|\hat{v}|ij\rangle_{AS},
\]
where we have removed a single-particle state $k\le F$, that is a state below the Fermi level.  




Calculating the difference 
\[
  E[\Phi^{\mathrm{HF}}(A)]-   E[\Phi^{\mathrm{HF}}(A-1)] 
  = \langle k | \hat{h}_0 | k \rangle +
  \frac{1}{2}\sum_{i=1;i\ne k}^A\langle ik|\hat{v}|ik\rangle_{AS} + \frac{1}{2}\sum_{j=1;j\ne k}^A\langle kj|\hat{v}|kj\rangle_{AS},
\]
which becomes 
\[
  E[\Phi^{\mathrm{HF}}(A)]-   E[\Phi^{\mathrm{HF}}(A-1)] 
  = \langle k | \hat{h}_0 | k \rangle +\sum_{j=1}^A\langle kj|\hat{v}|kj\rangle_{AS}
\]
which is just our definition of the Hartree-Fock single-particle energy
\[
  E[\Phi^{\mathrm{HF}}(A)]-   E[\Phi^{\mathrm{HF}}(A-1)] 
  = \epsilon_k^{\mathrm{HF}} 
\]




Similarly, we can now compute the difference (recall that the single-particle states $abcd > F$)
\[
  E[\Phi^{\mathrm{HF}}(A+1)]-   E[\Phi^{\mathrm{HF}}(A)]= \epsilon_a^{\mathrm{HF}}. 
\]
If we then recall that the binding energy differences 
\[
BE(A)-BE(A-1) \hspace{0.5cm} \mathrm{and} \hspace{0.5cm} BE(A+1)-BE(A), 
\]
define the separation energies, we see that the Hartree-Fock single-particle energies can be used to
define separation energies. We have thus our first link between nuclear forces (included in the potential energy term) and an observable quantity defined by differences in binding energies. 




We have thus the following interpretations (if the single-particle fields do not change)
\[
BE(A)-BE(A-1)\approx  E[\Phi^{\mathrm{HF}}(A)]-   E[\Phi^{\mathrm{HF}}(A-1)] 
  = \epsilon_k^{\mathrm{HF}}, 
\]
and
\[
BE(A+1)-BE(A)\approx  E[\Phi^{\mathrm{HF}}(A+1)]-   E[\Phi^{\mathrm{HF}}(A)] =  \epsilon_a^{\mathrm{HF}}. 
\]
If  we use ${}^{16}\mbox{O}$ as our closed-shell nucleus, we could then interpret the separation energy
\[
BE(^{16}\mathrm{O})-BE(^{15}\mathrm{O})\approx \epsilon_{0p^{\nu}_{1/2}}^{\mathrm{HF}}, 
\]
and
\[
BE(^{16}\mathrm{O})-BE(^{15}\mathrm{N})\approx \epsilon_{0p^{\pi}_{1/2}}^{\mathrm{HF}}.
\]




Similalry, we could interpret
\[
BE(^{17}\mathrm{O})-BE(^{16}\mathrm{O})\approx \epsilon_{0d^{\nu}_{5/2}}^{\mathrm{HF}}, 
\]
and 
\[
BE(^{17}\mathrm{F})-BE(^{16}\mathrm{O})\approx\epsilon_{0d^{\pi}_{5/2}}^{\mathrm{HF}}.
\]
We can continue like this for all $A\pm 1$ nuclei where $A$ is a good closed-shell (or subshell closure)
nucleus. Examples are ${}^{22}\mbox{O}$, ${}^{24}\mbox{O}$, ${}^{40}\mbox{Ca}$, ${}^{48}\mbox{Ca}$, ${}^{52}\mbox{Ca}$, ${}^{54}\mbox{Ca}$, ${}^{56}\mbox{Ni}$, 
${}^{68}\mbox{Ni}$, ${}^{78}\mbox{Ni}$, ${}^{90}\mbox{Zr}$, ${}^{88}\mbox{Sr}$, ${}^{100}\mbox{Sn}$, ${}^{132}\mbox{Sn}$ and ${}^{208}\mbox{Pb}$, to mention some possile cases.




We can thus make our first interpretation of the separation energies in terms of the simplest
possible many-body theory. 
If we also recall that the so-called energy gap for neutrons (or protons) is defined as
\[
\Delta S_n= 2BE(N,Z)-BE(N-1,Z)-BE(N+1,Z),
\]
for neutrons and the corresponding gap for protons
\[
\Delta S_p= 2BE(N,Z)-BE(N,Z-1)-BE(N,Z+1),
\]
we can define the neutron and proton energy gaps for ${}^{16}\mbox{O}$ as
\[
\Delta S_{\nu}=\epsilon_{0d^{\nu}_{5/2}}^{\mathrm{HF}}-\epsilon_{0p^{\nu}_{1/2}}^{\mathrm{HF}}, 
\]
and 
\[
\Delta S_{\pi}=\epsilon_{0d^{\pi}_{5/2}}^{\mathrm{HF}}-\epsilon_{0p^{\pi}_{1/2}}^{\mathrm{HF}}. 
\]






% --- begin exercise ---
\begin{doconceexercise}
\refstepcounter{doconceexercisecounter}

\subsection*{Exercise \thedoconceexercisecounter: Derivation of Hartree-Fock equations}


Consider a Slater determinant built up of single-particle orbitals $\psi_{\lambda}$, 
with $\lambda = 1,2,\dots,N$.

The unitary transformation
\[
\psi_a  = \sum_{\lambda} C_{a\lambda}\phi_{\lambda},
\]
brings us into the new basis.  
The new basis has quantum numbers $a=1,2,\dots,N$.
Show that the new basis is orthonormal.
Show that the new Slater determinant constructed from the new single-particle wave functions can be
written as the determinant based on the previous basis and the determinant of the matrix $C$.
Show that the old and the new Slater determinants are equal up to a complex constant with absolute value unity.
(Hint, $C$ is a unitary matrix).

\end{doconceexercise}
% --- end exercise ---




% --- begin exercise ---
\begin{doconceexercise}
\refstepcounter{doconceexercisecounter}

\subsection*{Exercise \thedoconceexercisecounter: Derivation of Hartree-Fock equations}


Consider the  Slater  determinant
\[
\Phi_{0}=\frac{1}{\sqrt{n!}}\sum_{p}(-)^{p}P
\prod_{i=1}^{n}\psi_{\alpha_{i}}(x_{i}).
\]
A small variation in this function is given by
\[
\delta\Phi_{0}=\frac{1}{\sqrt{n!}}\sum_{p}(-)^{p}P
\psi_{\alpha_{1}}(x_{1})\psi_{\alpha_{2}}(x_{2})\dots
\psi_{\alpha_{i-1}}(x_{i-1})(\delta\psi_{\alpha_{i}}(x_{i}))
\psi_{\alpha_{i+1}}(x_{i+1})\dots\psi_{\alpha_{n}}(x_{n}).
\]
Show that
\[
\langle \delta\Phi_{0}|\sum_{i=1}^{n}\left\{t(x_{i})+u(x_{i})
\right\}+\frac{1}{2}
\sum_{i\neq j=1}^{n}v(x_{i},x_{j})|\Phi_{0}\rangle=\sum_{i=1}^{n}\langle \delta\psi_{\alpha_{i}}|\hat{t}+\hat{u}
|\phi_{\alpha_{i}}\rangle
+\sum_{i\neq j=1}^{n}\left\{\langle\delta\psi_{\alpha_{i}}
\psi_{\alpha_{j}}|\hat{v}|\psi_{\alpha_{i}}\psi_{\alpha_{j}}\rangle-
\langle\delta\psi_{\alpha_{i}}\psi_{\alpha_{j}}|\hat{v}
|\psi_{\alpha_{j}}\psi_{\alpha_{i}}\rangle\right\}
\]

\end{doconceexercise}
% --- end exercise ---


% ------------------- end of main content ---------------


\printindex

\end{document}

